\documentclass[]{article}
\usepackage{lmodern}
\usepackage{amssymb,amsmath}
\usepackage{ifxetex,ifluatex}
\usepackage{fixltx2e} % provides \textsubscript
\ifnum 0\ifxetex 1\fi\ifluatex 1\fi=0 % if pdftex
  \usepackage[T1]{fontenc}
  \usepackage[utf8]{inputenc}
\else % if luatex or xelatex
  \ifxetex
    \usepackage{mathspec}
  \else
    \usepackage{fontspec}
  \fi
  \defaultfontfeatures{Ligatures=TeX,Scale=MatchLowercase}
\fi
% use upquote if available, for straight quotes in verbatim environments
\IfFileExists{upquote.sty}{\usepackage{upquote}}{}
% use microtype if available
\IfFileExists{microtype.sty}{%
\usepackage{microtype}
\UseMicrotypeSet[protrusion]{basicmath} % disable protrusion for tt fonts
}{}
\usepackage[margin=1in]{geometry}
\usepackage{hyperref}
\hypersetup{unicode=true,
            pdftitle={My first R Markdown Document},
            pdfauthor={Brad Boehmke},
            pdfborder={0 0 0},
            breaklinks=true}
\urlstyle{same}  % don't use monospace font for urls
\usepackage{graphicx,grffile}
\makeatletter
\def\maxwidth{\ifdim\Gin@nat@width>\linewidth\linewidth\else\Gin@nat@width\fi}
\def\maxheight{\ifdim\Gin@nat@height>\textheight\textheight\else\Gin@nat@height\fi}
\makeatother
% Scale images if necessary, so that they will not overflow the page
% margins by default, and it is still possible to overwrite the defaults
% using explicit options in \includegraphics[width, height, ...]{}
\setkeys{Gin}{width=\maxwidth,height=\maxheight,keepaspectratio}
\IfFileExists{parskip.sty}{%
\usepackage{parskip}
}{% else
\setlength{\parindent}{0pt}
\setlength{\parskip}{6pt plus 2pt minus 1pt}
}
\setlength{\emergencystretch}{3em}  % prevent overfull lines
\providecommand{\tightlist}{%
  \setlength{\itemsep}{0pt}\setlength{\parskip}{0pt}}
\setcounter{secnumdepth}{0}
% Redefines (sub)paragraphs to behave more like sections
\ifx\paragraph\undefined\else
\let\oldparagraph\paragraph
\renewcommand{\paragraph}[1]{\oldparagraph{#1}\mbox{}}
\fi
\ifx\subparagraph\undefined\else
\let\oldsubparagraph\subparagraph
\renewcommand{\subparagraph}[1]{\oldsubparagraph{#1}\mbox{}}
\fi

%%% Use protect on footnotes to avoid problems with footnotes in titles
\let\rmarkdownfootnote\footnote%
\def\footnote{\protect\rmarkdownfootnote}

%%% Change title format to be more compact
\usepackage{titling}

% Create subtitle command for use in maketitle
\newcommand{\subtitle}[1]{
  \posttitle{
    \begin{center}\large#1\end{center}
    }
}

\setlength{\droptitle}{-2em}

  \title{My first R Markdown Document}
    \pretitle{\vspace{\droptitle}\centering\huge}
  \posttitle{\par}
    \author{Brad Boehmke}
    \preauthor{\centering\large\emph}
  \postauthor{\par}
    \date{}
    \predate{}\postdate{}
  

\begin{document}
\maketitle

\hypertarget{data-transformation}{%
\section{Data Transformation}\label{data-transformation}}

We like to use dplyr for data transformation.

\hypertarget{filtering}{%
\subsection{Filtering}\label{filtering}}

If I want to filter my data, use \texttt{filter()}. Here is an example:

\begin{verbatim}
##     mpg cyl  disp  hp drat    wt  qsec vs am gear carb
## 1  22.8   4 108.0  93 3.85 2.320 18.61  1  1    4    1
## 2  21.4   6 258.0 110 3.08 3.215 19.44  1  0    3    1
## 3  24.4   4 146.7  62 3.69 3.190 20.00  1  0    4    2
## 4  22.8   4 140.8  95 3.92 3.150 22.90  1  0    4    2
## 5  32.4   4  78.7  66 4.08 2.200 19.47  1  1    4    1
## 6  30.4   4  75.7  52 4.93 1.615 18.52  1  1    4    2
## 7  33.9   4  71.1  65 4.22 1.835 19.90  1  1    4    1
## 8  21.5   4 120.1  97 3.70 2.465 20.01  1  0    3    1
## 9  27.3   4  79.0  66 4.08 1.935 18.90  1  1    4    1
## 10 26.0   4 120.3  91 4.43 2.140 16.70  0  1    5    2
## 11 30.4   4  95.1 113 3.77 1.513 16.90  1  1    5    2
## 12 21.4   4 121.0 109 4.11 2.780 18.60  1  1    4    2
\end{verbatim}

\hypertarget{select}{%
\subsection{Select}\label{select}}

We can use the \texttt{select()} function to select specific variables
of interest

\begin{verbatim}
##     mpg cyl  disp
## 1  22.8   4 108.0
## 2  21.4   6 258.0
## 3  24.4   4 146.7
## 4  22.8   4 140.8
## 5  32.4   4  78.7
## 6  30.4   4  75.7
## 7  33.9   4  71.1
## 8  21.5   4 120.1
## 9  27.3   4  79.0
## 10 26.0   4 120.3
## 11 30.4   4  95.1
## 12 21.4   4 121.0
\end{verbatim}

\hypertarget{visualization}{%
\section{Visualization}\label{visualization}}

We like to use \texttt{ggplot2} for visualization

\includegraphics{first-rmarkdown_files/figure-latex/unnamed-chunk-3-1.pdf}


\end{document}
